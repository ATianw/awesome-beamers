% !TEX TS-program = xelatex 
% +bibtex+xelatex
% !TEX encoding = UTF-8
% !Mode:: "TeX:UTF-8"
% +-----------------------------------------------------------------------------
% | File: resume-zh
% | Author: huxuan
% | E-mail: i(at)huxuan.org
% | Created: 2012-12-18
% | Last modified: 2013-03-16
% | Description:
% |     A Chinese Resume Example in LaTeX based on resumecls
% |
% | Copyrgiht (c) 2012-2013 by huxuan. All rights reserved.
% +-----------------------------------------------------------------------------

\documentclass[color]{resumecls}
\providecommand{\LyX}{L\kern-.1667em\lower.25em\hbox{Y}\kern-.125emX\@}
\ctexset{today=small}

\setmainfont{Arial}
% \setsansfont{TeX Gyre Schola} 
\setmonofont{Consolas}
 
\name{黄湘云}
\organization{中国矿业大学(北京)}
\address{北京市海淀区学院路丁11号,100083}
\mobile{+86 188 1097 2907}
\mail{xiangyunfaith@outlook.com}
\homepage{https://xiangyunhuang.github.io/}
% \leftfooter{最后更新: \today}
% \rightfooter{\url{http://example.com/resume-zh.pdf}}

\begin{document}

\begin{table}

\maketitle

%%%%%%%%%%%%%%%%%%%%%%%%%%%%%%%%%%%%%%%%%%%%%%%%%%%%%%%%%%%%%%%%%%%%%%%%%%%%%%%
\heading{教育经历}
\entry{2em}{Xrp{8em}}{%
    \heiti{中国矿业大学(北京)} & 北京市 & 2015.09-2018.06 \\
}
\entry{4em}{lXX}{%
    理学学术型硕士学位 & 理学院 & 统计学专业 \\
}
\entry{4em}{lXX}{%
    研究方向:数据分析与统计计算 \\
}
\entry{4em}{lXX}{%
    毕业论文:空间广义线性混合效应模型及其应用 \\
}
\entry{2em}{Xrp{8em}}{%
    \heiti{中国矿业大学(北京)} & 北京市 & 2011.09-2015.06 \\
}
\entry{4em}{lXX}{%
    理学学士学位 & 理学院 & 数学与应用数学专业 \\
}
% \entry{4em}{lXX}{%
    % 专业课程: 数学分析, 高等代数, 空间解析几何, C 程序设计, 数值分析, 数学建模, \\
	% \qquad 概率论与统计统计, 时间序列分析, 应用随机过程, 多元统计分析, 数据结构与算法等 \\
% }
\entry{4em}{lXX}{%
    毕业论文:统计方法在区域经济社会发展综合评价中的应用 \\
}
%%%%%%%%%%%%%%%%%%%%%%%%%%%%%%%%%%%%%%%%%%%%%%%%%%%%%%%%%%%%%%%%%%%%%%%%%%%%%%%
% \heading{科研经历}
% \entry{2em}{Xp{8em}}{%
    % \heiti{地点} & 起止时间 \\
% }
% \entry{4em}{X}{实验室名称 \quad 职位}
% \entry{6em}{X}{%
    % 研究方向和具体内容 \\
    % 发表成果(亦可使用bibtex,像这样\cite{label},见文档最后注释内容) \\
% }

%%%%%%%%%%%%%%%%%%%%%%%%%%%%%%%%%%%%%%%%%%%%%%%%%%%%%%%%%%%%%%%%%%%%%%%%%%%%%%%
\heading{工作经历}
\entry{2em}{Xp{8em}}{%
    \heiti{新浪} & 2017.09-2017.11 \\
}
\entry{4em}{X}{BIP平台部 \quad 数据分析师}
\entry{6em}{X}{%
    负责新浪新闻客户端日志分析,主要使用SQL 从ClickHouse 数据仓库提取数据,R 语言分析和可视化,并完成日报,用数据分析协助其他部门决策,如服务器资源调度等。 \\
}

%%%%%%%%%%%%%%%%%%%%%%%%%%%%%%%%%%%%%%%%%%%%%%%%%%%%%%%%%%%%%%%%%%%%%%%%%%%%%%%
\heading{校园经历}
\entry{2em}{Xp{8em}}{%
    % 院团委宣传部干事 			& 2011.10--2012.06 \\
    校红会学生分会组织部长 		& 2012.06--2013.06 \\
	% 校红十字学生分会组织的爱心支教活动 & 2012.09--2012.10 \\
	大学生科研创新训练项目组长 & 2012.09--2014.12 \\
	CDA 数据分析师培训 			& 2016.04--2016.06 \\
	% 北京大学高维统计短期课程 	& 2016.06 \\
	参与国家自然科学基金项目(编号:11671398) 		 & 2017.01--2020.12 \\
	研究生上机助教(数学建模、数值分析、运筹学等课程)  & 2015.09--2016.12
}

%%%%%%%%%%%%%%%%%%%%%%%%%%%%%%%%%%%%%%%%%%%%%%%%%%%%%%%%%%%%%%%%%%%%%%%%%%%%%%%
\heading{资格证书及获奖情况}
\entry{2em}{Xr}{%
    全国大学英语 \quad 四级(448)和六级(462) & 2011.06--2014.06 \\
	第29届全国部分地区物理竞赛  \quad 三等奖 & 2012 \\
    2013 年高教社杯全国大学生数学建模竞赛 \quad 北京市二等奖 & 2013.12 \\
	% 2011-2012学年本科生优秀学生 \quad 三等奖学金 & 2012 \\
	2013-2014学年本科生优秀学生 \quad 二等奖学金 & 2013 \\
	% 校科技文化节科普创作大赛 \quad 优秀奖 & 2013 \\
	% 校物理实验竞赛 \quad 一等奖 & 2013 \\
	2015-2016 年度研究生优秀学生 \quad 一等奖学金 & 2016 \\
	2016-2017 年度研究生优秀学生荣誉称号 & 2017 \\
}

%%%%%%%%%%%%%%%%%%%%%%%%%%%%%%%%%%%%%%%%%%%%%%%%%%%%%%%%%%%%%%%%%%%%%%%%%%%%%%%
\heading{专业技能}
\entry{2em}{lX}{%
    精通 & R、ggplot2、SparkR \\
    熟练 & Markdown、Docker、Git/Github、Matlab \\
    掌握 & \LaTeX{}、\LyX{}、RStudio\\
    使用 & Python (SymPy/NumPy)、HTML/CSS/JS (Blog) \\
	% 语言: & R 、SQL、LATEX、MATLAB、Python
	% 系统: & Windows、Linux \\
	% 软件: & RStudio、GitHub、\LaTeX{}、\LyX{}、Markdown、Docker、ggplot2、SparkR
	% 统计:& 数据清洗、探索、处理、建模和可视化、统计计算和绘图\\
	% 英语: & 能够流畅地阅读专业领域内的论文和其他文献,并熟练地用英文检索遇到的问题\\
}

\heading{项目实战}
\entry{2em}{lX}{%
	R语言社区开发者贡献关系网络可视化\\
	分析目标:可视化R 语言社区中开发者相互合作的关系网络、挖掘对社区贡献大的组织、个人\\
	主要工作:抓取CRAN 官方11000+ R 包信息、RStudio 官网下载日志,
			正则表达式提取开发者、贡献者 \\ 
		\quad 字段,清洗后,建立有向图网络,使用ggplot2等可视化包,呈现贡献网络
}

%%%%%%%%%%%%%%%%%%%%%%%%%%%%%%%%%%%%%%%%%%%%%%%%%%%%%%%%%%%%%%%%%%%%%%%%%%%%%%%
\heading{社区贡献}
\entry{2em}{lX}{%
    统计之都成员 & 主站作者、中文论坛版主、审稿人、编辑 \\
    已发表的主站文章:R 语言做符号计算(2016.07.08) & 随机数生成及其在统计模拟中的应用(2017.05.26) \\
	漫谈条形图(2017.10.15)
}

%%%%%%%%%%%%%%%%%%%%%%%%%%%%%%%%%%%%%%%%%%%%%%%%%%%%%%%%%%%%%%%%%%%%%%%%%%%%%%%
% 如果不需要发表成果,注释这一段即可
% \heading{附:发表成果}
% \vspace{-6em}
% \bibliography{resume}

%%%%%%%%%%%%%%%%%%%%%%%%%%%%%%%%%%%%%%%%%%%%%%%%%%%%%%%%%%%%%%%%%%%%%%%%%%%%%%%
\end{table}
\end{document}
