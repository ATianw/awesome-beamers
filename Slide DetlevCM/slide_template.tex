% !TEX TS-program = xelatex
% !TEX encoding = UTF-8 Unicode

\documentclass[10pt,xcolor=x11names,UTF8]{ctexbeamer}
\mode<presentation>
{
  \usetheme{DetlevCM}
% JLTree cuerna  epyt  detlevcm  metropolis  verona  PhnomPenh 
% fibeamer  hohenheim   nirma

% fibeamer/beamerthemefibeamer
% UR Jacobs TorinoTh
\usecolortheme{rose}
\useinnertheme[shadow]{rounded}  
\useoutertheme[subsection=false]{smoothbars}
  \setbeamercovered{transparent}
 % \usetheme{Singapore}
  % \usecolortheme[named=SpringGreen4]{structure}
}

\usepackage[english]{babel}
% or whatever

\usepackage[utf8]{inputenc}
% or whatever

% \usepackage{times}
% \usepackage[T1]{fontenc}


\title[Short Paper Title] % (optional, use only with long paper titles)
{Presentation Title}

\subtitle
{Presentation Subtitle} % (optional)

\author[Author, Another] % (optional, use only with lots of authors)
{F.~Author\inst{1} \and S.~Another\inst{2}}
% - Use the \inst{?} command only if the authors have different
%   affiliation.

\institute[Universities of Somewhere and Elsewhere] % (optional, but mostly needed)
{
  \inst{1}%
  Department of Computer Science\\
  University of Somewhere
  \and
  \inst{2}%
  Department of Theoretical Philosophy\\
  University of Elsewhere}
% - Use the \inst command only if there are several affiliations.
% - Keep it simple, no one is interested in your street address.

\date[Short Occasion] % (optional)
{Date / Occasion}

\subject{Talks}

% \pgfdeclareimage[height=0.5cm]{university-logo}{university-logo-filename}
% \logo{\pgfuseimage{university-logo}}

% Delete this, if you do not want the table of contents to pop up at
% the beginning of each subsection:
\AtBeginSubsection[]
{
  \begin{frame}<beamer>{Outline}
    \tableofcontents[currentsection,currentsubsection]
  \end{frame}
}

%\beamerdefaultoverlayspecification{<+->}


\begin{document}

\begin{frame}
  \titlepage
\end{frame}

\begin{frame}{Outline}
  \tableofcontents
  % You might wish to add the option [pausesections]
\end{frame}


\section{Introduction}

\subsection[Short First Subsection Name]{First Subsection Name}

\begin{frame}{Make Titles Informative. Use Uppercase Letters.}{Subtitles are optional.}
  % - A title should summarize the slide in an understandable fashion
  %   for anyone how does not follow everything on the slide itself.

  \begin{itemize}
  \item
    Use \texttt{itemize} a lot.
  \item
    Use very short sentences or short phrases.
  \end{itemize}
\end{frame}

\begin{frame}{Make Titles Informative.}

  You can create overlays\dots
  \begin{itemize}
  \item using the \texttt{pause} command:
    \begin{itemize}
    \item
      First item.
      \pause
    \item    
      Second item.
    \end{itemize}
  \item
    using overlay specifications:
    \begin{itemize}
    \item<3->
      First item.
    \item<4->
      Second item.
    \end{itemize}
  \item
    using the general \texttt{uncover} command:
    \begin{itemize}
      \uncover<5->{\item
        First item.}
      \uncover<6->{\item
        Second item.}
    \end{itemize}
  \end{itemize}
\end{frame}


\subsection{Second Subsection}

\begin{frame}{Make Titles Informative.}
\end{frame}

\begin{frame}{Make Titles Informative.}
\end{frame}



\section*{Summary}

\begin{frame}{Summary}

  % Keep the summary *very short*.
  \begin{itemize}
  \item
    The \alert{first main message} of your talk in one or two lines.
  \item
    The \alert{second main message} of your talk in one or two lines.
  \item
    Perhaps a \alert{third message}, but not more than that.
  \end{itemize}
  
  % The following outlook is optional.
  \vskip0pt plus.5fill
  \begin{itemize}
  \item
    Outlook
    \begin{itemize}
    \item
      Something you haven't solved.
    \item
      Something else you haven't solved.
    \end{itemize}
  \end{itemize}
\end{frame}


\end{document}


